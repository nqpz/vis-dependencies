\documentclass[a4paper,12pt]{article}

\title{\textbf{Visualization}: Exploring library dependencies in modern software projects\\\emph{Midway report}}
\author{Ask Neve Gamby, Rune Filtenborg Hansen,\\
Henrik Urms, and Niels Gustav Westphal Serup}

\usepackage[T1]{fontenc}
\usepackage{lmodern}
\usepackage[utf8]{inputenc}
\usepackage[british]{babel}
\usepackage{microtype}
\usepackage{underscore}
\usepackage{amsmath}
\usepackage{graphicx}
\usepackage[hidelinks]{hyperref}

\setlength{\parskip}{1ex}
\setlength{\parindent}{0pt}
\setlength{\parfillskip}{30pt plus 1 fil}

\begin{document}

\maketitle

\section{Introduction}

This project is about getting an overview of the structure of depended-upon code
in a large software project, and help developers make informed decisions about
what software libraries (not) to use.


\subsection{Related Work}

We have looked at a series of articles to determine what could quantitatively
describe a ``good'' software library.

TODO, include the ones we already found? write a bit about them
We have looked at some different papers, looking at visualizing sofware dependencies.
Their main goal is to discern the dependencies from classes and objects. The papers assert
that the typical representation of depencies in software is made using graph structure.
The articles try to present new ways of representing such graphs. [Paper1] uses a specific
sort of tree to filter the amount of nodes down to a smaller subset such that it becomes
clearer from a user perspective to understand the visualization. This is important as the
number of nodes can become overwhelming, though we suffer some information loss using this
method. [Paper2] propeses a new graph constructing language, to help creating visualizations
showing graphs. This new graph language is argued to be more efficient in both layout and
setup, as it can take several arguments at a time, instead of specifying each argument.
This method uses different channels to encode the properties of the dependency graph it
shows. It uses color to map different categories and size to mark the usage of these.

\section{Dataset}

Our datasets include the approximately 2000 Haskell programming language
packages on the Stackage dependency manager at \url{https://www.stackage.org/}.

For each package it is possible to get the following attributes:

\begin{itemize}
\item What other packages the package depends on.
\item How many other packages depend on this package.
\item How much this package \emph{uses} each of its depended-upon
packages.
\item How long the package has been in development.
\item How many public releases there are of this package.
\item In some cases it is also possible to see how many people have contributed
to the package, although this is only the case if the package includes a link to
its source management system page, e.g. a GitHub page.
\end{itemize}


\section{Tasks}

\subsection{Target Group}

Our target group can be describes as developers of modern software, more
specifically developers interested in improving whatever project they are
working on.  This especially includes:

\begin{itemize}
\item Developers having done lots of code sprints over a period of months, and
now want to take more thorough look at the code as a whole.
\item Developers taking over as maintainers of a new project in need of
refactoring.
\end{itemize}


\subsection{Domain}

To aid these developers, it would be nice to show them all kinds of stats about
a software project.  To limit ourselves, we have chosen to only show the effects
of the dependencies.

Using the attributes of each package in the dataset, we would like to show our
definition of a healthy package, and how much each dependency is used.

We can also show conflicts between dependencies.  This should not be a problem
for packages on Stackage, but it might happen for local, untested projects where
packages are added and removed rapidly during development.  This is also solved
with conventional command-line tools, but can be nice to show in a visualization
as well to get a more complete picture.


\section{Design}

We have included a design sketch in figure~\ref{fig:sketch0}.

\begin{figure}[h!]
\begin{center}
\includegraphics[width=0.8\textwidth]{sketch0.pdf}
\caption{A sketch of the visualization.}
\label{fig:sketch0}
\end{center}
\end{figure}

In this sketch, the main package ``futhark'' that we want to inspect is at the
top, and its dependencies are at the bottom.  Each package has a color
describing its maturity.  When you click on the package, the attributes
determining the maturity (as described in the previous sections) should be
expanded and described.

Instead of lines with varying sizes, we could also vary the sizes of the the
dependency boxes, and not care that much whether they are located perfectly
below the main package.  This might make the visualization easier to implement
and still provide enough visual clues.

The visualization could also be expanded to show conflicts between dependencies
as e.g. dotted lines, since this is also useful knowledge.


\section{A Scenario}

TODO, think of something, probably something with many developers and a messy
development strategy, and then a refactor


\section{Implementation}

\subsection{Data extraction}

I'm working on this --Niels

\subsection{Visualization coding}

We expect to program the visualization in D3.js.

TODO: Do we?


\end{document}
